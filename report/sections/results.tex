\section{Experimental Results}

\subsection{Quantitative Results}

The model was evaluated on the test set using three standard metrics:

\begin{itemize}
    \item Mean Absolute Error (MAE): Average absolute difference between predicted and ground-truth counts
    \item Root Mean Square Error (RMSE): Square root of mean squared error, penalizing larger deviations more heavily
    \item Average Accuracy (AA): $\text{AA} = \frac{1}{N}\sum_{i=1}^{N}\left(1 - \frac{|GT_i - MP_i|}{GT_i}\right)$, where higher values indicate better performance
\end{itemize}

All per-image statistics were computed on the set of matched images (147 images).

\begin{table}[h]
    \centering
    \caption{Performance Comparison (147 test images, mean GT count: 63.38)}
    \begin{tabular}{@{}lcccc@{}}
        \toprule
        \textbf{Model}            & \textbf{MAE} & \textbf{RMSE} & \textbf{AA} & \textbf{MAPE (\%)} \\
        \midrule
        Our implementation        & 1.3715       & 1.7689        & 0.9777      & 2.23               \\
        \midrule
        Original Paper (test set) & 22.0         & 32.3          & 0.9696      & ---                \\
        \bottomrule
    \end{tabular}
\end{table}

\begin{itemize}
    \item The implementation achieves MAE = 1.3715 chickens ($\approx$1.37) and AA = 0.9777 (97.77\%), indicating high counting accuracy suitable for practical applications on this dataset
    \item RMSE = 1.7689 shows the model handles most images with errors $<$2--3 chickens
    \item For images with 60--70 chickens per frame (mean GT count: 63.38), the relative MAE is 0.0216 ($\approx$2.16\%), and MAPE is 2.23\%, which demonstrates commercial viability for monitoring/counting applications
    \item Comparison with original paper: The implementation achieves comparable (slightly better) AA performance (0.9777 vs. 0.9696 reported in paper), despite differences in dataset characteristics. The original paper evaluated on images with significantly higher chicken densities (200--300+ chickens per image based on visualizations in their Figure 4), compared to our dataset's mean of 63.38 chickens per image. This explains the absolute MAE/RMSE differences: their MAE of 22.0 chickens on 200--300 chicken scenes corresponds to $\approx$7--11\% relative error, while our MAE of 1.37 on 60--70 chicken scenes represents $\approx$2\% relative error. Key dataset differences include: (1) scene density: 3--5$\times$ more chickens per frame in original paper, (2) dataset size: paper used 10 test images (from 56 total), we used 147 test images, (3) farm environments: different capture conditions and sources
\end{itemize}

\subsection{Error Distribution}

Per-image error statistics from \texttt{stats\_per\_image.csv}:

\begin{table}[h]
    \centering
    \caption{Error Distribution Statistics}
    \begin{tabular}{@{}lc@{}}
        \toprule
        \textbf{Metric}          & Value  \\
        \midrule
        Mean absolute error      & 1.3715 \\
        Std. deviation of errors & 1.1172 \\
        Median absolute error    & 1.0998 \\
        90th percentile error    & 2.7552 \\
        Max absolute error       & 5.8610 \\
        \bottomrule
    \end{tabular}
\end{table}

The standard deviation of per-image absolute errors (1.12) and maximum error (5.86) indicate generally consistent performance with a small number of harder examples; the bulk of errors remain small (median abs error $\approx$1.10).

\subsection{Qualitative Results}

Inference outputs include:
\begin{itemize}
    \item Density maps: Heatmap visualizations (saved as PNG) show predicted density distributions. High-density regions correctly correspond to chicken locations
    \item Spatial coherence: Total Variation regularization produces smooth, natural-looking density maps without spurious peaks
    \item Count accuracy: Visual inspection confirms density map integrals closely match ground-truth counts
\end{itemize}

The model showcases its ability to handle:

\begin{enumerate}
    \item Dense clustering with heavy occlusion
    \item Illumination variations (bright/shadowed areas)
    \item Scale variations (chickens at different distances)
    \item Background clutter (feeders, equipment, flooring patterns)
\end{enumerate}
